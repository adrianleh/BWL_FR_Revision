\documentclass[a4paper]{article}
\usepackage[utf8]{inputenc}
\usepackage[T1]{fontenc}
\usepackage[light,condensed,math]{kurier}
\usepackage[ngerman]{babel}
\usepackage{ntheorem}
\usepackage{graphicx}
\usepackage{floatrow}
\usepackage{float}
\usepackage{hyperref}
\usepackage{mathtools}
\usepackage{amssymb}
\theoremstyle{break}

\newtheorem{formula}{Formel}[section]
\newtheorem{defi}{Definition}[section]
\newtheorem{ann}{Bemerkung}[section]
\newtheorem{der}{Folgerung}[section]
\newtheorem{ex}{Beispiel}[section]
\newtheorem{why}{Vorteile}[section]
\newtheorem{whynot}{Nachteile}[section]

\author{Adrian E. Lehmann}
\newcommand{\ms}{$\mu$-$\sigma$}
\newcommand{\msd}{\ms-Diagramm}
\newcommand{\mbd}{$\mu$-$\beta$-Diagramm}
\title{BWL FR: Teil Finanzwirtschaft: Zusammenfassung}
\begin{document}
	\maketitle
	\tableofcontents
	\newpage
	
\section{Einführung}

\begin{defi}{Investition}
	Eine \textbf{Investition} ist eine \textbf{Entscheidung}, die zunächst eine \textbf{Auszahlung zur Folge} hat
\end{defi}

\begin{defi}{Finanzierung}
	\textbf{Finanzierung} ist eine \textbf{Entscheidung}, die zunächst eine \textbf{Einzahlung zur Folge} hat
\end{defi}
\begin{ann}
	Bei Finanzierung ist zwischen \textbf{Eigenkapital}, welches aus \textbf{Firmen- und Eigentümerbeständen} stammt und \textbf{Fremdkapital}, welches von \textbf{Dritten} stammt.
\end{ann}

\section{Capital Budgeting}
\begin{defi}{Barwert (present value)}
	Wert den zukünftige Zahlungen nun besitzen.\\
	Bei Zinssatz $r$ und Zahlung im Höhe $x$ in $t$ Perioden gilt :\\
	$$BW =  \frac{x}{(1+r)^t} $$
\end{defi}
\begin{defi}{Kaptialwert (Net present value)}
	Der Kaptialwert ist bezieht im Vergleich zum Barwert die \textbf{Anafangszahlung} (AAZ) ein. 
	$$ KW = -AAZ + BW$$
\end{defi}
\begin{defi}{Kapitalwertmethode}
	Eine Investion ist profitabel $\boldmath{	\Leftrightarrow KW > 0}$. \\Die Kapitalwertmethode sagt, in diesem Fall soll die Investion getätigt werden
\end{defi}
\begin{why}{Kapitalwertmethode}
	\begin{itemize}
		\item Zahlungen als Grundlage
		\item Mehrperiodizität
		\item Berücksichtigung riskioadäquater Renditne
	\end{itemize}
\end{why}
\begin{whynot}{Kapitalwertmethode}
	\begin{itemize}
		\item Vergleichbarkeit
	\end{itemize}
\end{whynot}
\begin{defi}{Kaptialwertrate}
	$$KWR = \frac{BW}{AAZ}$$
	Sagt aus, wie das Verhältnis Einahmen $\leftrightarrow$ Ausgaben ist.
	Investion profitabel, wenn KWR $> 1$
\end{defi}
\begin{why}{Kaptialwertrate}
	\begin{itemize}
		\item Vergleichbarkeit
	\end{itemize}
\end{why}
\begin{defi}{Amortisationsrechnung}
	Bestimme Zeit, bis AAZ erreicht
	$$ t_a = \min_{t \in \mathbb{N}_0}{(\sum_{i=0}^{t}{BW_i} - AAZ > 0)} $$
	Selektiere Projekte, die bis zum Planungshorizont armortisiert sind
\end{defi}
\begin{why}
	\begin{itemize}
		\item Zugeschnitten auf Zeithorizont
	\end{itemize}
\end{why}
\begin{whynot}
	\begin{itemize}
		\item Gesamtrendite nicht brücksichtigt
		\item Zahlungen nach $t_a$ nicht brücksichtigt
	\end{itemize}
\end{whynot}
\begin{defi}{Interne Zinssatzmethode (IZSM)}
	Löse:
	$$ -AAZ + \sum_{i=0}^{t} \frac{c_i}{(1 + z)^i} = 0 $$
	, wobei  t Anzahl der Perioden, $c_i$ Zahlung in Periode i, $z$ IZS.
	Investiere, wenn $z > r$
\end{defi}
\begin{why}{IZSM}
	\begin{itemize}
		\item Vergleichbarkeit
		\item Mehrperiodig
	\end{itemize}
\end{why}
\begin{why}{IZSM}
	\begin{itemize}
		\item $n$ Vortzeichenwechsel $\Rightarrow$ $n$  Lösungen
		\item Bei verschiedenen Größen nicht sauber vergleich bar $\Rightarrow$ Differenzprojekt
	\end{itemize}
\end{why}
\section{Aktien}
\subsection{IPO}
Schritte eines IPOs:
\begin{enumerate}
	\item Auswahl Emissionsbank
	\item Festsetzung der Transaktionscharakteristika
	\item Due Diligence
	\item Entwicklung Equity Story
	\item Road Show / Bookbuilding
	\item Preisfestsetzung
	\item Zuteilung
	\item Erster Handelstag
	\item Kurspflege
	\item Lockup Periode
\end{enumerate}
\begin{ann}{IPO Phänomene}
	\item \textbf{Underpricing}: Bookbuildingprice < Handelskurs
	\item \textbf{Zyklizität}: In guten Zeiten mehr IPOs (Business Cycle)
	\item \textbf{Hohe Kosten}
	\item \textbf{Performace langfristig meist mittelmäßig}
\end{ann}
\subsection{Dividende}
Dividende sind Auszahlungen des EKs an Eigentümer.\\
Bei Auszahlung einer Divdende um $x$ Geldeinheiten, fällt der Aktienkurs auch um $x$ Geldeinheiten
\begin{ann}
	Dividendenerhöhung signalisiert positive Erwartung und sorgt für Kurssteigerung. (Senkung analog)
\end{ann}
\begin{ann}{Dividend smoothing}
	Dividenden werden konservativ angepasst, da Kürzung vermieden werden sollte
\end{ann}
\section{Portfeuilletheorie}

\begin{formula}{Halteperio de Rendite}

$$t_h =  \prod_{i=0}^{t}{(1-r_i)} - 1$$
\end{formula}
\begin{formula}{Geometrisch-durschnittliche Rendite}
	$$ (1+r_g)^t = \prod_{i=0}^{t}{(1-r_i)}$$
\end{formula}
\begin{formula}{Arithmetisch-durschnittliche Rendite}
	$$ r_a = \frac{1}{n} \cdot \sum_{i=0}^{t}{r_i}$$
\end{formula}
Folgende Formeln aus: \url{https://github.com/adrianleh/investments_revision/}
\subsection{Markowitz}
 \begin{defi}[Portfolio]
	Ein Portfolio ist eine gewichtete Zusammenstellung von Wertpapieren. Dies ist als Vektor $w = (w_1, \dots, w_n)$ modellierbar, wobei $w_i$ das Gewicht des $i$-ten Wertpapier darstellt
\end{defi}
\begin{defi}[Leerverkaufsbeschränkung]
	Sei $w = (w_1, \dots, w_n)$ PF, dann gilt unter LVK-Beschr: $w_i \geq 0 ~ ~ \forall i$
\end{defi}
\begin{defi}[\msd]
	Ein \msd stellt die Verbindung zwischen Rendite und Risiko dar. Normalerweise wird dies durch eine Hyperbel dargestellt.
\end{defi}
\begin{defi}[Dominanz]
	Seien WP1, WP2 Wertpapiere mit erw. Renditen $\mu_1$ und $\mu_2$ sowie Risiken $\sigma_1$ und $\sigma_2$\\
	Dann gilt: \textbf{WP1 dominiert WP2} $\Longleftrightarrow ~ \mu_1 > \mu_2 \wedge \sigma_1 < \sigma_2$\\
	\emph{Aus WP1 dominiert WP2 folgt \textbf{NICHT}, dass WP2 im Portfolio nicht sinnvoll ist ($\rightarrow$ Korrelation)}
\end{defi}
\begin{defi}[GVMP]
	Das GVMP ist das Portfolio $w$, wo $\sigma$ minimal. Dies ist der Scheitel der Hyperbel in einem \msd 
\end{defi}
\begin{ann}
	Das GVMP ist unter LVK-Beschr erreichbar $\Longleftrightarrow ~ \rho_{12} < min \{\sigma_1, \sigma_2\}$
\end{ann}
\begin{defi}[Effiziente PFs]
	Ein PF $w$ ist effizient $\Longleftrightarrow ~ \nexists$ PF $\tilde{w}: \tilde{w}$ dominiert $w$
\end{defi}
\begin{defi}[Optimales Portfolio]
	Punkt im \msd in dem die Indifferenzkurve des Investors ein effizientes PF erreicht
\end{defi}
\begin{defi}[Erreichbare Portfolios]
	Die erreichbaren Portfolios befinden sich in der Fläche, welche durch die Hyperbel im \msd begrenzt wird.
	Die Portfolios auf der Hyperbel heißen auch ``Randportfolios''.
\end{defi}
\begin{der}
	Jedes Randportfolio ist LK von von 2 weiteren Randportfolios:\\
	$\forall$ RandPF $p ~ \exists$ RandPF $g, h ~ \exists \lambda_g, \lambda_h \in \mathbb{R}: p = \lambda_g g + \lambda_h h$
\end{der}
\begin{formula}{Rendite Aktie}
	$$ \tilde{r} = \frac{\tilde{S}(1)+D-\tilde{S}(0)}{\tilde{S}(0)}$$
\end{formula}
\begin{formula}[Erw. PF-Rendite]
	$$\mu_w = \sum_{i=1}^{n}w_i\cdot\mu_i$$
\end{formula}
\begin{formula}[PF-Varianz] 
	$$\sigma^{2}_w = \sum_{i=1}^{n}\sum_{j=1}^{n} w_i \cdot w_j \cdot Cov(\tilde{r}_i, \tilde{r}_j)$$
	$$=  \sum_{i=1}^{n}\sum_{j=1}^{n} w_i \cdot w_j \cdot \rho_{ij} \cdot \sigma_i \cdot \sigma_j$$
\end{formula}
Sei $n = 2$
\begin{formula}
	$$w_1 = 1 - w_2$$
\end{formula}
\begin{formula}[Erw.-Wert]
	$$\mu_w = w_1 \cdot \mu_1 + w_2 \cdot \mu_2$$
\end{formula}
\begin{formula}[Varianz]
	$$\sigma^{2}_w = w^{2}_1 \cdot \sigma^{2}_1 + w^{2}_2 \cdot \sigma^{2}_2 + 2 \cdot w_1 \cdot w_2 \cdot \sigma_1 \cdot \sigma_2 \cdot \rho_{12}$$
	$$\Rightarrow \sigma^{2}_w = w^{2}_1 \cdot \sigma^{2}_1 + (1-w_1)^{2} \cdot \sigma^{2}_2 + 2 \cdot w_1 \cdot (1-w_1) \cdot \sigma_1 \cdot \sigma_2 \cdot \rho_{ij}$$
\end{formula}
\subsection{CAPM}

\textbf{Annahmen:}
\begin{itemize}
	\item Investoren optimieren auf $\mu$-$\sigma$ Basis
	\item Investoren haben homogene Erwartungen
	\item Der Anlagehorizont aller Investoren ist gleich
	\item Kapitalmärkte sind vollkommen und friktionslos
	\item Es besteht risikolose Anlage- und Aufnahmemöglichkeit zum selben Zinssatz $r$ und in bel. Höhe
	\item Märkte bestehen aus handelbaren WPs mit festem Angebot
\end{itemize}

\textbf{Input:}
\begin{itemize}
	\item Erwartungswerte über zukünftige Kurse
\end{itemize}

\textbf{Output:}
\begin{itemize}
	\item Heutige Kurse passen sich bis zur Markträumung an
\end{itemize}

\textbf{Stärken CAPM:}
\begin{itemize}
	\item Relativ einfaches linears Gleichgewichtsmodeel
	\item Plausible Annahmen
	\item Robust gegenüber aufweichen der Annahmen:
	\begin{itemize}
		\item LVK Beschränkung ändert GGW Modell nicht
		\item Risikoloses Instrument: Setze $\beta = 0$        
	\end{itemize}
\end{itemize}

\textbf{Schwächen CAPM:}
\begin{itemize}
	\item Empirisch schwer zu testen (Rolls Kritik):
	\begin{itemize}
		\item Marktportfolio unmöglich zu erreichen
		\item Lineare Risiko/Rendite Beziehung ist mathematische Tautologie
	\end{itemize}
\end{itemize}

Die Idee ist, dass jeder Investor selbst sich Aktien und risikoloses Instrument. Dies erzeugt ein \textbf{Gleichgewicht}: Der Markt ist geräumt und alle Investoren sind befriedigt. \textbf{$\Longrightarrow$ Jeder Investor hat gleiches Portfolio (genannt: ``Marktportfolio'')}

\begin{defi}{Kaptialmarktlinie - KML}
	Alle effiziente PFs liegen auf der KML (Gerade).\\
	Das einzige eff. reine Aktienportfolio ist das Marktportfolio
\end{defi}

\begin{defi}{Marktpreis des Risikos}
	$$\frac{\mu_M - r}{\sigma_M}$$
\end{defi}

\begin{der}
	Aktive Handelsstrategie NICHT sinnvoll, sondern nur das Nachbilden des Marktportfolios    
\end{der}

\begin{ann}[Risiko]
	Risiken können in 2 Klassen unterteilt werden:\\
	\begin{enumerate}
		\item Systematisches Risiko (nicht div.bar): Risikobeitrag zum Gesamtrisiko $cov(\tilde{r}_j, \tilde{r}_M)$
		\item Unsystematisches Risiko (div.bar): Unternehmensbez. Risiko $\sigma^{2}_{\epsilon_j}$
	\end{enumerate}    
\end{ann}

\begin{defi}[$\beta_j$]
	$$\beta_j \coloneqq \frac{cov(\tilde{r}_j, \tilde{r}_M)}{\sigma^{2}_M}$$
\end{defi}

Sei $w$ WP und $M$ Markt bestehend aus $w^M_i, i \in \{1, \dots, N\}$\\
Es gilt: $\beta_i = \frac{cov(\tilde{r}_i, \tilde{r}_M)}{\sigma^{2}_M}$
\begin{formula}[KML]
	$$\mu_w = r + \frac{\mu_M - r}{\sigma_M} \cdot \sigma_w$$
\end{formula}
\begin{formula}[WPML]
	$$\mu_j = r + (\mu_M - r) \cdot \beta_i$$
\end{formula}
\begin{formula}[Risiko Marktportfolio]
	$$\sigma^{2}_M = \sum_{i=1}^{N}w_i^M \cdot cov(\tilde{r}_i, \tilde{r}_M)$$
\end{formula}


\end{document}